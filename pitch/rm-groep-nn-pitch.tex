%==============================================================================
% Research Methods - Group assignment - pitch
%==============================================================================

\documentclass[aspectratio=169]{beamer}

% You can increase the font size a bit with the option `14pt'

%==============================================================================
% Preamble
%==============================================================================

%---------- Layout ------------------------------------------------------------

\usetheme{hogent}

% Choose the background color theme, hgwhite (white background) or hgblack
% (black background)
\usecolortheme{hgwhite}

%---------- Packages ----------------------------------------------------------
% You can add additional packages if you need them.

\usepackage[dutch]{babel}      % Dutch language support (remove or change to
                               % "english" for English)

\usepackage{booktabs}          % Better looking tables
\usepackage{multirow,multicol} % Merge table cells
\usepackage{eurosym}           % Euro symbool

%---------- Metadata ----------------------------------------------------------

\title{Pitch onderzoeksvoorstellen.}
\subtitle{Research Methods, 25-26, groep NN}
\author{Erdener Bayramoglu \and Clara Bekaert \and Sanya Golovina \and Bavo Van Ginderen}
\date{\today}

%==============================================================================
% Presentation content
%==============================================================================

\begin{document}

{

    % An image cut out in the shape of a letter is a typical design element in
    % HOGENT's corporate identity. You can replace the image (in this case
    % "blurred-background-cellphone-coffee-842554.jpg") and the letter (in this
    % case "R") with your own choices. The image should be placed in the
    % `graphics` folder of your project and can be a jpeg or png.
    \setbeamertemplate{background}[imgletter]
        {blurred-background-cellphone-coffee-842554.jpg}{R}

    \begin{frame}
        \maketitle
    \end{frame}
}

\begin{frame}
    \frametitle{Inhoud.}

    \tableofcontents
\end{frame}

\section{Titel voorstel 1.}

\begin{frame}
    \frametitle{Inleiding.}

    Some text
\end{frame}

{
\setbeamertemplate{background}[imgletter]
    {blurred-background-cellphone-coffee-842554.jpg}{I}

\begin{frame}

    {\huge \textbf{Voorbeeld van tekst door een afbeelding}}

\end{frame}
}

\subsection{Subsectie 1.1.}

\begin{frame}
    \frametitle{Titel.}

    \begin{itemize}
        \item Lijn één
        \item Lijn twee
        \item Lijn drie
    \end{itemize}
\end{frame}

\subsection{Subsectie 1.2.}

\begin{frame}
    \frametitle{Twee kolommen.}

    \begin{columns}[c]
        \column{.5\textwidth}
        \begin{itemize}
            \item Lijn 1
            \item Lijn 2
            \item Lijn 3
        \end{itemize}

        \column{.5\textwidth}
        \begin{itemize}
            \item Lijn 1
            \item Lijn 2
            \item Lijn 3
        \end{itemize}
    \end{columns}
\end{frame}

\begin{frame}
    \frametitle{Kleuren}

    % These are examples of the colours defined in HOGENT's corporate identity.
    \textbf{\textcolor{hgdarkgreen}{tekst}
    \textcolor{hgpink}{tekst}
    \textcolor{hgochre}{tekst}
    \textcolor{hgorange}{tekst}
    \textcolor{hgpurple}{tekst}
    \textcolor{hgblue}{tekst}
    \textcolor{hglightgreen}{tekst}
    \textcolor{hgbrown}{tekst}
    \textcolor{hggrey}{tekst}
    \textcolor{hgyellow}{tekst}}

    \textbf{\colorbox{hgdarkgreen}{tekst}
    \colorbox{hgpink}{tekst}
    \colorbox{hgochre}{tekst}
    \colorbox{hgorange}{tekst}
    \colorbox{hgpurple}{tekst}
    \colorbox{hgblue}{tekst}
    \colorbox{hglightgreen}{tekst}
    \colorbox{hgbrown}{tekst}
    \colorbox{hggrey}{tekst}
    \colorbox{hgyellow}{tekst}}
\end{frame}

\section{Titel voorstel 2.}

\begin{frame}
\frametitle{Het tweede deel.}

\begin{table}
    \begin{tabular}{lll}
    \toprule
                    & \textbf{lorem ipsum} & \textbf{lorem ipsum} \\
    \midrule
    dolor sit amet & $3.255$              & $3.255$ \\
    \midrule
    lorem ipsum    & $425,43 (13,07\%)$   & $425,43 (13,07\%)$ \\
    \midrule
    dolor sit amet & $0,00$               & $1.273,31$ ($45\%$ -- gemiddeld)\\
    \midrule
    \textbf{{\small Verhouding kost}} & \textbf{$1,53$} & \textbf{$2,72$}\\
    \bottomrule
    \end{tabular}

    \label{tab:voorbeeld}
    \caption{Deze tabel heeft een bijschrift}
    \end{table}
\end{frame}

\subsection{Sectie 2.1.}

\begin{frame}
\frametitle{Afbeeldingen.}
    \begin{figure}
        \caption{Deze afbeelding heeft een bijschrift.}

        \includegraphics[height=.8\textheight]
            {beverage-coffee-computer-984536.jpg}
        % Bron: https://www.pexels.com/photo/hand-on-cup-of-coffee-984536/
        \label{img:voorbeeld}
    \end{figure}
\end{frame}

% If the HOGENT logo in the bottom right corner of the slide conflicts with
% the content, you can hide it by adding the option [plain] to the frame, as
% shown below.
\begin{frame}[plain]
    \includegraphics[width=\textwidth]{beverage-coffee-computer-984536.jpg}
\end{frame}

{
\setbeamertemplate{background}[imgletter]%
  {beverage-coffee-computer-984536.jpg}{T}

\begin{frame}

    {\huge \textbf{This is a famous quote that says something about
    my point I want to make.}}

    \bigskip

    \textbf{Name Quotemaker}

\end{frame}
}

\begin{frame}[fragile]
    \texttt{monogespatieerd lettertype met ligaturen -> <= >=}

    \begin{verbatim}
        a <- 1
        b <- 2
        a + b
    \end{verbatim}
\end{frame}

\begin{frame}{Wiskundige formule}
  \begin{eqnarray}
    X_{t} = \alpha x_{t} + (1-\alpha)(X_{t-1} + b_{t-1}) & 0 \leq \alpha \leq 1 \\
    b_{t} = \beta(X_{t}-X_{t-1}) + (1-\beta)b_{t-1} & 0 \leq \beta \leq 1 
    \label{eq:doubleSmoothing}
  \end{eqnarray}
\end{frame}

\end{document}
